\title{Perimeter Compression in self-healing swarms}
\author{
        Brockway M, Eliot N, Kendal D, \\
        Hexham University\\
        Cronkley Campus\\
        Department of Computer Science\\
}
\date{\today}

\documentclass[12pt,a4paper]{article}
\usepackage[margin=0.65in]{geometry}
\usepackage{graphicx}
\usepackage{float}
\usepackage{hyperref}
\usepackage{relsize}
\usepackage{amsmath}
  {
%	\theoremstyle{plain}
	\newtheorem{assumption}{Assumption}
}
\usepackage[most]{tcolorbox}
\tcbset{textmarker/.style={%
		enhanced,
		parbox=false,boxrule=0mm,boxsep=0mm,arc=0mm,
		outer arc=0mm,left=6mm,right=3mm,top=7pt,bottom=7pt,
		toptitle=1mm,bottomtitle=1mm,oversize}}

\newtcolorbox{importantBox}{textmarker,
	borderline west={6pt}{0pt}{red},
	colback=red!10!white}

\newcommand{\important}[1]{\begin{importantBox} \textbf{Important:} #1 \end{importantBox}}
\newcommand{\magn}[1]{\Vert{#1}\Vert}
\newcommand{\card}[1]{\vert{#1}\vert}

\begin{document}
\maketitle

\begin{abstract}
Perimeter Compression is a technique where by a void reducing effect can be added to a basic swarming algorithm. The affect is dependant upon perimeter identification and is controlled by applying two factors to the existing swarming formula. One to the cohesion calculation and the other to the repulsion calculation.
\end{abstract}

\section{Introduction}
Perimeter compression is a technique that creates a ``pull'' effect between perimeter agents. It is dependant upon perimeter agent identification as discussed by Eliot et. al. in \cite{eliot2017methods, eliot2018metric, eliot2019void}.
\paragraph{}
The aim of the algorithm is to reduce the spacing between perimeter-based agents by reducing the repulsion field (Figure. \ref{fig:stableswarm}) and increasing the cohesion affect on perimeter agents. $S_b$ is the sensor field. $O_b$ is the obstacle field. $C_b$ is the cohesion field. $R_b$ is the repulsion field. The implementation involves introducing two controlling factors; $p_c$ (Perimeter Compression Cohesion) which increases the cohesion vector ($k_c\rightarrow p_ck_c$) and $p_r$ (Perimeter Compression Repulsion) which reduces the size of the repulsion field ($k_r\rightarrow p_rk_r$) on the inter-perimeter agents.
\begin{assumption}
	$p_c >= 1$
\end{assumption}
\begin{assumption}
	$p_r <= 1$
\end{assumption}
\begin{figure}[H]
	\centering
	\includegraphics[width=0.5\linewidth]{figures/stableswarm}
	\caption[Agent Fields]{Agent Fields}
	\label{fig:stableswarm}
\end{figure}

\section{Resultant Vector Calculation}\label{resultantVector}
In the Original work by Eliot et. al. the resultant vector of an agent was calculated using Equation~\ref{eq:resultantVector}. Where $k_c,k_r,k_d,k_o$ are weighting factors for the summed vectors associated with each interaction. The new algorithm requires each individual agent to have a variation to the vector generated inside each calculation based on the perimeter status of the agent and each neighbour. The equation has been simplified (Eq.~\ref{eq:newResultantVector}) and the weighting factors have been transposed into the calculations along with additional weighting factors that are applied to specific agents within the cohesion and repulsion vector calculations.

\begin{equation}\label{eq:resultantVector}
	v(b) = k_cv_c(b) + k_rv_r(b) + k_dv_d(b) + k_ov_o(b)
\end{equation}
\begin{equation}\label{eq:newResultantVector}
	v(b) = v_c(b) + v_r(b) + v_d(b) + v_o(b)
\end{equation}

The effect of these weighting factors can be seen in Figure~\ref{fig:compressioneffect1}. The metric used in producing the graph is based upon the inter-agent magnitudes~\cite{eliot2018metric}.

\begin{figure}[H]
	\centering
	\includegraphics[width=0.75\linewidth]{figures/CompressionEffect1}
	\caption[Compression Effect]{Compression Effect based on Magnitude change}
	\label{fig:compressioneffect1}
\end{figure}

\section{Repulsion}\label{repulsion}
\subsection{Repulsion neighbours}\label{repulsion:neighbours}
The repulsion component of an agent's movement is calculated from its interaction with its neighbours $n_r(b)$ (Eq.~\ref{eq:repulsion1}) that are within the agent's ($b$) repulsion field ($R_b$) or the adjusted ($p_r$) range ($\mathsf{erf}(b,b')$) if the agents are both perimeter-based. Equation~\ref{eq:repulsion2} shows the calculation of the effective field. As the agent's repulsion field is always within the cohesion field (Eq.~\ref{eq:cohesion1}), the repulsion neighbours can also be defined as a subset of the cohesion neighbours $n_c(b)$~(Eq.~\ref{eq:repulsion3}).\\

\begin{equation}\label{eq:repulsion1}
n_r(b) = \{b' \in \mathcal{S} : b \neq b' \land \lVert\vec{b b'}\rVert \leq \mathsf{erf}(b,b')\}
\end{equation}

\begin{equation}\label{eq:repulsion2}
\mathsf{erf}(b, b') = \mathsf{if} \;
\mathsf{per}(b) \; \mathsf{and} \; \mathsf{per}(b') \; \mathsf{then} \;
p_rR_b \; \mathsf{else} \; R_b
\end{equation}

\begin{equation}\label{eq:repulsion3}
n_r(b) = \{b' \in n_c(b)~:~\magn{bb'} <= \mathsf{erf}(b,b')\}
\end{equation}

\subsection{Repulsion compression}\label{repulsion:compression}
To reduce the repulsion effect the compression weighting ($p_r$) is applied to an agent's repulsion field if both itself and it's neighbour are perimeter agents. The function $\mathsf{per}()$ returns an agent's perimeter status. An agent is identified as a perimeter agent using the technique shown by Eliot et.al. in \cite{eliot2019void} which uses a cyclic-check of neighbour agent angles to identify ``gaps'' in the neighbours.

If the condition of both agents being a perimeter is met ($\mathsf{per}(b) \; \mathsf{and} \; \mathsf{per}(b')$) the repulsion field distance is multiplied by the compression factor ($p_r$) and the new field effect is used to generate a resultant-repulsion-vector (Eq.~\ref{eq:repulsion5}). 

The effect of Equation \ref{eq:repulsion4} will be to reduce the repulsion of inter-perimeter-based agents allowing them to be closer together before a reduced repulsion-vector is applied. 

\important{The repulsion-vector that is generated is based upon $p_rR_b$, the reduced repulsion field, and not the full $R_b$ field. This is to scale the resultant-repulsion-vector as well as reducing the repulsion field.}

\begin{equation}\label{eq:repulsion4}
v_r(b) = \frac{1}{\lvert n_r(b)\rvert}\sum_{b' \in n_r(b)} k_r\left(\lVert\vec{b b'}\rVert - \mathsf{erf}(b,b') \, \right)\widehat{bb'}
\end{equation}

\begin{equation}\label{eq:repulsion5}
\mathsf{erf}(b, b') = \mathsf{if} \;
\mathsf{per}(b) \; \mathsf{and} \; \mathsf{per}(b') \; \mathsf{then} \;
p_rR_b \; \mathsf{else} \; R_b
\end{equation}

\section{Cohesion}\label{cohesion}
\subsection{Cohesion neighbours}\label{cohesion:neighbours}
The cohesion component of an agent is calculated in a similar way to the repulsion in that it is dependent upon the proximity of neighbours. Where $n_c(b)$ is the set of neighbour agents for $b$ (Eq. \ref{eq:cohesion1}). The inclusion of an agent from a swarm ($S$) in by the agent's cohesion field ($C_b$).

\begin{equation}\label{eq:cohesion1}
n_c(b) = \{b' \in S~:~b' \neq b \land\magn{bb'} <= C_b\}
\end{equation}

The affect of an agent being within this set is that it will generate a vector that should `encourage' agents to maintain their proximity. i.e. generate a cohesive swarm. The general weighted ($k_c$) formula for agents to maintain their proximity is to direct their motion towards the central point of all neighbouring agents as shown in Equation~\ref{eq:cohesion2}. This formula includes the $k_c$ quotient that allows the cohesion effect to be `balanced' with respect to other vector influences as described in ~\cite{eliot2017methods,eliot2018metric,eliot2019void} 

\begin{equation}\label{eq:cohesion2}
v_c(b) = \frac{1}{\lvert n_c(b)\rvert} \sum_{b' \in n_c(b)}k_c\vec{b b'}
\end{equation}

\subsection{Compression cohesion}
The cohesion component of the compression effect ($p_c$) is applied when an agent ($b$) and its neighbour ($b'$) are both perimeter-based. If the agents are not both perimeter-based then the agents vector is only scaled by $k_c$ (Eq.~\ref{eq:cohesion4}). The effect of the additional cohesion-compression weighting is to increases the size of the generated cohesion-vector $\mathsf{efc}(b,b')$ (Eq.~\ref{eq:cohesion3}). 

\begin{equation}\label{eq:cohesion3}
v_c(b) = \frac{1}{\lvert n_c(b)\rvert} \sum_{b' \in n_c(b)}\mathsf{ekc}(b, b')\, \vec{b b'}
\end{equation}

\begin{equation}\label{eq:cohesion4}
\mathsf{ekc}(b, b') = \mathsf{if} \; \mathsf{per}(b) \; \mathsf{and} \; \mathsf{per}(b') \; \mathsf{then} \; \mathrm{p}_ck_c \; \mathsf{else} \; k_c
\end{equation}

\section{Conclusions}\label{conclusions}
From the initial simulations it is possible to show that the technique is able to successfully remove voids and surround an obstacle as shown in the video \href{https://youtu.be/3eY1vvq0JWo}{https://youtu.be/3eY1vvq0JWo}.

\bibliographystyle{abbrv}
\bibliography{perimeter}

\end{document}