%\documentclass[12pt,a4paper]{article}
\documentclass[12pt,a4paper]{IEEEtran}
\usepackage[margin=0.65in]{geometry}
\usepackage[affil-it]{authblk}
\usepackage{graphicx}
\usepackage[table,xcdraw]{xcolor}
\usepackage{caption}
\usepackage{subcaption}
\usepackage{float}
\usepackage{hyperref}
\usepackage{relsize}
\usepackage{amsmath}
{
	%	\theoremstyle{plain}
	\newtheorem{assumption}{Assumption}
}
\usepackage[most]{tcolorbox}
\tcbset{textmarker/.style={%
		enhanced,
		parbox=false,boxrule=0mm,boxsep=0mm,arc=0mm,
		outer arc=0mm,left=6mm,right=3mm,top=7pt,bottom=7pt,
		toptitle=1mm,bottomtitle=1mm}}

\newtcolorbox{importantBox}{textmarker,
	borderline west={6pt}{0pt}{red},
	colback=red!10!white}

\newcommand{\important}[1]{\begin{importantBox} \textbf{Important:} #1 \end{importantBox}}
\newcommand{\magn}[1]{\Vert{#1}\Vert}
\newcommand{\card}[1]{\vert{#1}\vert}
%\newcommand{\vbb}[2]{\vec{#1#2}}
\newcommand{\vbb}[2]{#2-#1}

\title{Improved perimeter compression and swarm formation in self-healing swarms.}
\author[1,*]{Neil Eliot}
\author[2]{David Kendall}
\author[2]{Michael Brockway}
\affil[1] {Northumbria University, Faculty of Engineering and Environment, Department of Computer and Information Sciences}
\affil[2] {Hexham University, Faculty of Computer Science}
\affil[*] {Corresponding author: Dr Neil Eliot, neil.eliot@northumbria.ac.uk}
\date{\today}


\begin{document}
\maketitle

\begin{abstract}
Perimeter Compression is a technique where by a void reducing effect can be added to a basic swarming algorithm. The effect is dependant upon perimeter identification and is controlled by applying two weighting factors to the existing swarming formulae. One to the cohesion calculation and the other to the repulsion calculation.
\end{abstract}

\section{Introduction}
When cohesion and repulsion field effects (sometimes referred to as potential fields~\cite{BAF:06,eliot2018metric,VG:05,SW:03,Son2017,liang2019swarm}) are used to create a swarming effect, the stable structures that develop are limited to either straight edges or partial lattices \cite{eliot2017methods}. The maintenance of a well-structured swarm is crucial to effective deployment for applications such as reconnaissance or artificial pollination, where `blind spots' are best eliminated \cite{elamvazhuthi2015optimal}, and containment, where the swarm is used to surround an object or region \cite{cao2012distributed}. Over time swarms form regular shapes~\cite{RAZ:13} and perimeters form of partial lattices that may contain so-called \textit{anomalies}, such as concave `dents' or convex `peaks'~\cite{eliot2019void}. These anomalies contribute to the disruption of an otherwise well-structured swarm. The key, therefore, is to ensure that these \textit{anomalies} are dynamically removed from a swarm.\\
Perimeter compression is a technique that creates a `pull' effect between perimeter agents. It is dependant upon perimeter agent identification as discussed by Eliot et. al. in \cite{eliot2017methods, eliot2018metric, eliot2019void} and discussed in Section~\ref{sec:perimeterDetection}.\\
The aim of this new algorithm is to reduce the spacing between perimeter-based agents by reducing the repulsion field and increasing the cohesion effect on only the perimeter agents. Figure \ref{fig:stableswarm}) shows an agent and it fields. $S_b$ is the sensor field. $O_b$ is the obstacle field. $C_b$ is the cohesion field and $R_b$ is the repulsion field. The implementation involves introducing two controlling weights; $p_c$ (Perimeter Compression Cohesion) which increases the magnitude of the cohesion vector ($k_c\rightarrow p_ck_c$) and $p_r$ (Perimeter Compression Repulsion) which reduces the size of the repulsion field and also the repulsion magnitude ($k_r\rightarrow p_rk_r$) of the perimeter-based agents.
\begin{assumption}
	$p_c >= 1$
\end{assumption}
\begin{assumption}
	$p_r <= 1$
\end{assumption}
\begin{figure}[H]
	\centering
	\includegraphics[width=0.8\linewidth]{figures/stableswarm}
	\caption[Agent Fields]{Agent Fields}
	\label{fig:stableswarm}
\end{figure}

\section{Related work}

As far back as 1987 swarm theory has adopted the use of field effects/potential fields to coordinate agents~\cite{REY:87} and this has continued since then in an attempt to improve the structure of a swarm, coordinate obstacles avoidance, and improve navigation~\cite{BAFVM:06,BAF:06,BFV:07,BM:09,eliot2018metric,VG:05,HC:09,SW:03,Son2017}. A prototype framework for self-healing swarms was developed by Dai et al., which considered how to manage agent failure in hostile environments \cite{DHMRZ:06}. This was similar to work by Vassev and Hinchey, who modelled swarm movement using the ASSL (Autonomic System Specification Language) \cite{VH:09}. This technique was employed by NASA (US National Aeronautics and Space Administration) for use in asteroid belt exploration as part of their ANTS (Autonomous Nano Technology Swarm) project. However, this work is focused towards failure of an agent's internal systems, rather than on the removal of anomalies in a swarm distribution. This need for formation control is also discussed by Speck and Bucci with respect to the diverse applications of swarms and the need to control a swarms structure~\cite{8430773}.

In the context of swarm structure maintenance, Roach et al. focussed on the effects of sensor failure, and the impact that has on agent distribution \cite{RMT:15}. Lee and Chong identified the issue of concave edges within swarms in an attempt to create regular lattice formations \cite{GN:08}, and the main focus of their work is the dynamic restructuring of inter-agent formations. Ismail and Timmis demonstrated the use of \textit{bio-inspired} healing using \textit{granuloma formation}, a biological method for encapsulating an antigen \cite{IT:10}. They have also considered the effect failed agents can have on a swarm when traversing a terrain~\cite{TIBW:16}. 

This void reduction effect through perimeter compression is an extension of the work presented by Eliot et al. \cite{eliot2019void}, Ismail and Timmis \cite{IT:10,TIBW:16}, and on the work of McLurkin and Demaine on the detection of perimeter types \cite{mclurkin2009}. However, perimeter type identification requires a communications infrastructure to allow the perimeter angle to be calculated. Communications within swarm formations limits swarm sizes and introduces performance problems \cite{fu2020formation}. The technique employed in this paper does not explicitly require the identification of the perimeter type as it would limit the size of the swarm\cite{eliot2019void,GN:08} and is therefore a reduced algorithm to identify any perimeter.

\section{Basic swarming model}\label{basicModel}
In the Original work by Eliot et. al. the resultant vector of an agent was calculated using Equation~\ref{eq:resultantVector1a}. Where $k_c,k_r,k_d,k_o$ are weighting factors for the summed vectors associated with each interaction. i.e. $v_c$, $v_r$, $v_d$, $v_o$ for cohesion, repulsion, direction and object avoidance respectively. 

\begin{equation}\label{eq:resultantVector1a}
	v(b) = k_cv_c(b) + k_rv_r(b) + k_dv_d(b) + k_ov_o(b)
\end{equation}

Equation~\ref{eq:resultantVector1a} shows the movement vector as a linear combination of a cohesion vector $v_c$ tending to move $b$ towards its neighbours, a repulsion vector $v_r$ tending to move $b$ away from its neighbours, a direction vector  $v_d$ tending to move $b$ towards a goal, and a vector $v_o$ tending to steer it away from obstacles. $k_c, k_r, ...$ are the scalar coefficients of the the linear combination.

This paper does not consider goals or obstacles so we assume $k_d = k_o = 0$ and omit the third and fourth terms.

\subsection{Cohesion}\label{cohesion}
The cohesion component is calculated based on the proximity of neighbours. Where $n_c(b)$ is the set of neighbour agents for $b$ (Eq. \ref{eq:cohesion1}). The inclusion of an agent from a swarm ($S$) in by the agent's cohesion field ($C_b$).

\begin{equation}\label{eq:cohesion1}
n_c(b) = \{b' \in S~:~b' \neq b \land\magn{\vbb{b}{b'}} \leq C_b\}
\end{equation}

The effect of an agent being within this set is that it will generate a vector that should `encourage' agents to maintain their proximity. i.e. generate a cohesive swarm. The general weighted ($k_c$) formula for agents to maintain their proximity is to direct their motion towards the central point of all neighbouring agents as shown in Equation~\ref{eq:cohesion2}. Where $\card{n_c(b)}$ denotes the cardinality of $n_c(b)$. This formula includes the $k_c$ quotient that allows the cohesion effect to be `balanced' with respect to other vector influences as described in ~\cite{eliot2017methods,eliot2018metric,eliot2019void}. 

\begin{equation}\label{eq:cohesion2}
v_c(b) = \frac{1}{\card{n_c(b)}} \sum_{b' \in n_c(b)}(\vbb{b}{b'})
\end{equation}

\subsection{Repulsion}\label{repulsion:neighbours}
The repulsion component of an agent's movement is calculated from interaction with its neighbours $n_r(b)$ (Eq.~\ref{eq:repulsion1a}) in a swarm ($\mathcal{S}$) that are within the agent's ($b$) repulsion field ($R_b$).

\begin{equation}\label{eq:repulsion1a}
n_r(b) = \{b' \in \mathcal{S} : b \neq b' \land \card{\vbb{b}{b'}} \leq R_b)\}
\end{equation}

The repulsion is then calculated as the average of all the vectors created by the agent ($b$) to the neighbours ($b'$) (Eg.~\ref{eq:repulsion2a}) and its proximity ($\magn{\vbb{b}{b'}} - R_b$). Where $\card{n_r(b)}$ denotes the cardinality of $n_r(b)$.

\begin{equation}\label{eq:repulsion2a}
v_r(b) = \frac{1}{\card{n_r(b)}}\sum_{b' \in n_r(b)} \left(\magn{\vbb{b}{b'}} - R_b \, \right)\widehat{\vbb{b}{b'}}
\end{equation}
Here, $\widehat{\vbb{b}{b'}}$ denotes $\vbb{b}{b'}$ normalized to unit length.

\section{Perimeter detection}\label{sec:perimeterDetection}
For perimeter compression to be applied to a swarm the perimeter needs to be detected. A perimeter can be defined as a continuous `surface' of agents that are not enclosed by other agents. These  agents may form an outer ({\color{green}green}) or inner ({\color{red}red}) boundary ~(Fig.~\ref{fig:innerOuterPerimeters}).

\begin{figure}[H]
	\begin{center}
		\includegraphics[width=5cm]{figures/PerimeterBots1}
	\end{center}
	\caption{{\color{green}Outer} and {\color{red}inner} swarm perimeters. \label{fig:innerOuterPerimeters}}
\end{figure}

The detection process is achieved using a cyclic analysis of the agents that surround an agent~(Fig.~\ref{fig:neighbours2}). Ghrist et al. discusses a similar technique using sweep angles~\cite{ghrist2008surrounding} as does McLurkin et al~\cite{mclurkin2009}. 

\begin{figure}[H]
	\centering
	\includegraphics[width=0.8\linewidth]{figures/neighbours2}
	\caption[Agent neighbours]{Agent neighbours}
	\label{fig:neighbours2}
\end{figure}

The initial detection of the agents is based on the distance that each agent in the swarm is away from the current agent as described in Section~\ref{cohesion} and shown in Equation~\ref{eq:cohesion1}. The perimeter detection set is based on the azimuth angle ($\beta$) of each neighbour agent as shown in~Fig.~\ref{fig:neighbours2}.

The neighbour set $n_c(b)$ is then sorted in ascending order such that $\beta_0 < \beta_1 < \ldots < \beta_n$. Each consecutive pair of agents in the sequence now defines an \textit{edge} which has a length that can be calculated from their relative positions.

An agent is therefore on the perimeter of the swarm if it is not enclosed by the polygon defined in the sorted set $n_c(b)$. 

The polygon generated by $n_c(b)$ is considered to be `open' if two successive agents on the perimeter are separated by more than the size of the cohesion field ($C_b$).  

One further condition must be checked to detect a perimeter agent. When the polygon defined by $n_c(b)$ is closed it is possible that two or more neighbour agents are compressed to the point that they are within range $C_b$ but are `behind' agent $b$. This condition can be detected based on any neighbour pair angle being greater than $\pi$. Figure~\ref{fig:neighbours3} shows this condition with agent $b_0$ and $b_3$ assuming the neighbour agent pair are within range $C_b$. 

\begin{figure}[H]
	\centering
	\includegraphics[width=0.8\linewidth]{figures/neighbours3}
	\caption[Agent neighbours]{Agent neighbour angles}
	\label{fig:neighbours3}
\end{figure}

\section{Gap reduction model}
An enhancement to the basic model that has proven useful in combination with the enhancements detailed below, especially in reducing internal voids in the swarm, is to introduce another term on the right of Equation~\ref{eq:newModel2}. This gap reducing model is based upon the model described by Eliot et al. in~\cite{eliot2019void}.

For instance (omitting obstacles, destinations for now),
\begin{equation}\label{eq:newModel3}
v(b) = v_c(b) + v_r(b) + v_g(b)
\end{equation}

where $v_g(b)$ is a vector which will, in the case that a gap has been identified in the perimeter test for $b$, will contribute a motion of $b$ toward the midpoint of the gap.

Suppose that $n_r(b) = \{b_0, b_1, ... b_{n-1}\}$ are the neighbours of $b$ in azimuth angle order, and let the \emph{first} gap found in the perimeter test be $b' = b_i$ to $b'' = b_{(i+1)\%n}$. Then $k_g$ can be defines as Equation~\ref{eq:gap}. 

\small\begin{equation}\label{eq:gap}
v_g(b) = k_g \left(\frac{b' + b''}{2} - b \right) = k_g \frac{\left(\left(\vbb{b}{b'}\right) + \left(\vbb{b}{b''}\right)\right)}{2} 
\end{equation}\normalsize

$k_g$ is a weighting for the gap-filling vector allowing the combination of it with the other motion vectors (cohesion, repulsion, ...) to be ``tuned''.

\section{Perimeter compression models}
In this section we will discuss two applications of the $p_r$, $p_c$ and $p_{kr}$ compression parameters. The new algorithms requires each individual agent to modify the generated movement vector based upon the perimeter status of the agent and each neighbour. The final part of the section will introduce the addition of the gap reduction to the perimeter controls which effects a stabilisation property to the swarm perimeter. The basic perimeter control equation is shown in Equation~\ref{eq:newModel2}). The cohesion and repulsion weighting factors ($k_c$,$k_r$,$p_{kr}$) have been transferred into the original Equations \ref{eq:cohesion3}, \ref{eq:cohesion4}, \ref{eq:repulsion1}, \ref{eq:repulsion1-1} and, \ref{eq:repulsion2} so Equation~\ref{eq:resultantVector1a} now reads, simply,

\begin{equation}\label{eq:newModel2}
v(b) = v_c(b) + v_r(b)
\end{equation}

\subsection{Inter perimeter agent control $p_r$ and $p_c$}

The basic method of implementing perimeter compression is to reduce the repulsion filed to allow agents to move more closely to each other before they are repelled and to overcome some of the inner agent repulsion forcing the perimeter agents apart the cohesion weighting is increased.

\subsubsection{Modified cohesion model ($p_c$)}

The cohesion component of the compression effect ($p_c$) is applied when an agent ($b$) and its neighbour ($b'$) are both perimeter-based. If the agents are not both perimeter-based then the agents cohesion vector is scaled by $k_c$ (Eq.~\ref{eq:cohesion4}). The effect of the additional cohesion-compression weighting is to increase the size of the effective, generated cohesion-vector $\mathsf{ekc}(b,b')$ (Eq.~\ref{eq:cohesion3}) which causes the agent to have a tendency to move towards that neighbour. 

\begin{equation}\label{eq:cohesion3}
v_c(b) = \frac{1}{\lvert n_c(b)\rvert} \sum_{b' \in n_c(b)}\mathsf{ekc}(b, b')\, \vec{bb'}
\end{equation}

\small
\begin{equation}\label{eq:cohesion4}
\mathsf{ekc}(b, b') = \mathsf{if} \; \mathsf{per}(b) \; \mathsf{and} \; \mathsf{per}(b') \; \mathsf{then} \; \mathrm{p}_ck_c \; \mathsf{else} \; k_c
\end{equation}
\normalsize

\subsubsection{Modified repulsion model ($p_r$)}\label{repulsion:compression}
The repulsion compression component of the perimeter is applied by adjusted the \textbf{e}ffective \textbf{r}epulsion \textbf{f}ield ($\mathsf{erf}(b,b')$) if the agents are both perimeter-based. Equation~\ref{eq:repulsion4} shows the new formula to calculate the adjusted repulsion. As the agent's repulsion field is always within the cohesion field (Eq.~\ref{eq:repulsion1}), the repulsion neighbours can also be defined as a subset of the cohesion neighbours $n_c(b)$~(Eq.~\ref{eq:repulsion1-1}).\\

\small\begin{equation}\label{eq:repulsion1}
n_r(b) = \{b' \in \mathcal{S} : b \neq b' \land \magn{\vbb{b}{b'}} \leq \mathsf{erf}(b,b')\}
\end{equation}

\begin{equation}\label{eq:repulsion1-1}
n_r(b) = \{b' \in n_c(b)~:~\magn{\vbb{b}{b'}} \leq \mathsf{erf}(b,b')\}
\end{equation}\normalsize

Where $\mathsf{erf}$ is defined as:

\small\begin{equation}\label{eq:repulsion2}
\mathsf{erf}(b, b') = \mathsf{if} \;
\mathsf{per}(b) \; \mathsf{and} \; \mathsf{per}(b') \; \mathsf{then} \;
p_rR_b \; \mathsf{else} \; R_b
\end{equation}\normalsize

An agent is identified as a perimeter agent using the technique described in \S~\ref{sec:perimeterDetection} and shown by Eliot et.al. in \cite{eliot2019void} which uses a cyclic-check of neighbour agent angles to identify `gaps' in the neighbours.

If the condition of both agents being a perimeter is met ($\mathsf{per}(b) \; \mathsf{and} \; \mathsf{per}(b')$) the repulsion field distance is multiplied by the compression factor ($p_r$) and the new field effect is used to generate a resultant-repulsion-vector (Eq.~\ref{eq:repulsion2}). 

The effect of Equation \ref{eq:repulsion4} will be to reduce the repulsion of inter-perimeter-based agents allowing them to be closer together before a reduced repulsion-vector is applied. 

\small
\begin{equation}\label{eq:repulsion4}
v_r(b) = \frac{1}{\lvert n_r(b)\rvert}\sum_{b' \in n_r(b)} k_r\left(\lVert\vec{bb'}\rVert - \mathsf{erf}(b,b') \, \right)\widehat{bb'}
\end{equation}
\normalsize
%\small
%\begin{equation}\label{eq:repulsion5}
%\mathsf{erf}(b, b') = \mathsf{if} \;
%\mathsf{per}(b) \; \mathsf{and} \; \mathsf{per}(b') \; \mathsf{then} \;
%p_rR_b \; \mathsf{else} \; R_b
%\end{equation}
%\normalsize

\subsubsection{Combined effects of $p_c$ and $p_r$}

The effect of introducing the additional weighting factors ($p_c$ and $p_r$) can be seen in \S~\ref{sec:ExperimentalResults} which demonstrates the additional internal disturbance of the swarm caused by the compression algorithm and the removal of internal swarm voids. 

Figure~\ref{fig:voidRemoval} shows how the compression effect can remove a void from a swarm by surround an obstacle in a similar manner to the method described in~\cite{eliot2019void}.

\begin{figure}
	\centering
	\begin{subfigure}{0.4\textwidth}
		\centering	
		\includegraphics[width=1.0\linewidth]{figures/voidRemoval1}
		\caption[Void removal start]{Void removal start}
		\label{fig:voidRemovalStart}
	\end{subfigure}
	\begin{subfigure}{0.4\textwidth}
		\centering
		\includegraphics[width=1.0\linewidth]{figures/voidRemoval2}
		\caption[Void removal finish]{Void removal finish}
		\label{fig:voidRemovalFinish}
	\end{subfigure}
	\caption{Void removal through perimeter compression}
	\label{fig:voidRemoval}
\end{figure}

\subsection{Inner compression model}

\subsubsection{Inner + Gap compression model}

\subsection{Outer compression model}

\subsubsection{Outer compression model}


\section{Experimental results\label{sec:ExperimentalResults}}
\subsection{Baseline}
For all the experiments the parameters used to create the basic swarming effect are shown in Table~\ref{tab:swarmingEffect}. Where $C_b$ is the cohesion field, $k_c$ is the cohesion weighting, $R_b$ is the repulsion field, $k_r$ is the repulsion weighting and $k_g$ is the weighting factor applied in the comparison of the gap reduction algorithm discussed in~\cite{eliot2019void}. The swarm consists of 200 agents which are distributed with a void at the centre. These initial parameters create a hexagonal-based distribution of agents that stabilise as shown in Figure~\ref{fig:baselineSwarm}. This basic swarm is used as the initial state for all the experiments. 
 \begin{table}[h]
	\centering
	\tiny
	\begin{tabular}{|c|r|}
		\hline
		\rowcolor[HTML]{000000} 
		{\color[HTML]{FFFFFF} Swarming Variable} & {\color[HTML]{FFFFFF} Value} \\ \hline
		$C_b$ & \texttt{3.00} \\ \hline
		$k_c$ & \texttt{0.15}  \\ \hline
		$R_b$ & \texttt{2.00} \\ \hline
		$k_r$ & \texttt{50.00} \\ \hline
		$k_g$ & \texttt{25.00} \\ \hline
	\end{tabular}
  	\caption{Swarming effect parameters}
  	\label{tab:swarmingEffect}
\end{table}

\begin{figure}[H]
	\begin{center}
		\includegraphics[width=5cm]{figures/exp1Start}
	\end{center}
	\caption{Baseline swarm in stabilised configuration. \label{fig:baselineSwarm}}
\end{figure}

When the simulation is ran with no compression the changes are identified using a magnitude-based metric~\cite{eliot2018metric}. The resultant magnitudes generated are shown in figure~\ref{fig:baselineSwarmMagnitude}. These states are used as the baseline for the experiments to measure the effects of the compression algorithm and compare the new algorithm to the existing void reduction algorithm.

\begin{figure}[H]
	\begin{center}
		\includegraphics[width=8cm]{figures/interagentMagnitudeBaseline}
	\end{center}
	\caption{Baseline swarm in stabilised configuration. \label{fig:baselineSwarmMagnitude}}
\end{figure}

\subsection{Compression Effects\label{sec:CompressionEffect}}

The compression effect parameters are shown in tables~\ref{tab:compressionExperimentEffect1} and~\ref{tab:compressionExperimentEffect2}

\begin{table}[h]
	\centering
	\tiny
\begin{tabular}{ | >{\columncolor[HTML]{000000}}l | l | l | l | l | l | }
	\hline
	\rowcolor[HTML]{000000} 
	{\color[HTML]{FFFFFF} Pr/Pc} & {\color[HTML]{FFFFFF}10} & {\color[HTML]{FFFFFF}20} & {\color[HTML]{FFFFFF}30} & {\color[HTML]{FFFFFF}40} & {\color[HTML]{FFFFFF}50} \\ \hline
	    {\color[HTML]{FFFFFF}0.1} & 0.1/10 & 0.1/20 & 0.1/30 & 0.1/40 & 0.1/50  \\ \hline
		{\color[HTML]{FFFFFF}0.2} & 0.2/10 & 0.2/20 & 0.2/30 & 0.2/40 & 0.2/50  \\ \hline
		{\color[HTML]{FFFFFF}0.3} & 0.3/10 & 0.3/20 & 0.3/30 & 0.3/40 & 0.3/50  \\ \hline
		{\color[HTML]{FFFFFF}0.4} & 0.4/10 & 0.4/20 & 0.4/30 & 0.4/40 & 0.4/50  \\ \hline
		{\color[HTML]{FFFFFF}0.5} & 0.5/10 & 0.5/20 & 0.5/30 & 0.5/40 & 0.5/50  \\ \hline
		{\color[HTML]{FFFFFF}0.6} & 0.6/10 & 0.6/20 & 0.6/30 & 0.6/40 & 0.6/50  \\ \hline
		{\color[HTML]{FFFFFF}0.7} & 0.7/10 & 0.7/20 & 0.7/30 & 0.7/40 & 0.7/50  \\ \hline
		{\color[HTML]{FFFFFF}0.8} & 0.8/10 & 0.8/20 & 0.8/30 & 0.8/40 & 0.8/50  \\ \hline
		{\color[HTML]{FFFFFF}0.9} & 0.9/10 & 0.9/20 & 0.9/30 & 0.9/40 & 0.9/50  \\ \hline
	\end{tabular}
	\caption{Experiment parameters 1}
	\label{tab:compressionExperimentEffect1}
\end{table}

\begin{table}[h]
	\centering
	\tiny
\begin{tabular}{ | >{\columncolor[HTML]{000000}}l | l | l | l | l | l | }
	\hline
	\rowcolor[HTML]{000000} 
	{\color[HTML]{FFFFFF} Pr/Pc} & {\color[HTML]{FFFFFF}60} & {\color[HTML]{FFFFFF}70} & {\color[HTML]{FFFFFF}80} & {\color[HTML]{FFFFFF}90} & {\color[HTML]{FFFFFF}100}   \\ \hline
		{\color[HTML]{FFFFFF}0.1} & 0.1/60 & 0.1/70 & 0.1/80 & 0.1/90 & 0.1/100 \\ \hline
		{\color[HTML]{FFFFFF}0.2} & 0.2/60 & 0.2/70 & 0.2/80 & 0.2/90 & 0.2/100 \\ \hline
		{\color[HTML]{FFFFFF}0.3} & 0.3/60 & 0.3/70 & 0.3/80 & 0.3/90 & 0.3/100 \\ \hline
		{\color[HTML]{FFFFFF}0.4} & 0.4/60 & 0.4/70 & 0.4/80 & 0.4/90 & 0.4/100 \\ \hline
		{\color[HTML]{FFFFFF}0.5} & 0.5/60 & 0.5/70 & 0.5/80 & 0.5/90 & 0.5/100 \\ \hline
		{\color[HTML]{FFFFFF}0.6} & 0.6/60 & 0.6/70 & 0.6/80 & 0.6/90 & 0.6/100 \\ \hline
		{\color[HTML]{FFFFFF}0.7} & 0.7/60 & 0.7/70 & 0.7/80 & 0.7/90 & 0.7/100 \\ \hline
		{\color[HTML]{FFFFFF}0.8} & 0.8/60 & 0.8/70 & 0.8/80 & 0.8/90 & 0.8/100 \\ \hline
		{\color[HTML]{FFFFFF}0.9} & 0.9/60 & 0.9/70 & 0.9/80 & 0.9/90 & 0.9/100 \\ \hline
	\end{tabular}
	\caption{Experiment parameters 2}
	\label{tab:compressionExperimentEffect2}
\end{table}

\begin{figure}[H]
	\begin{center}
		\includegraphics[width=8cm]{figures/Experiment1}
	\end{center}
	\caption{Experiment Start \label{fig:experiment1}}
\end{figure}

\begin{figure}[H]
	\begin{center}
		\includegraphics[width=8cm]{figures/Experiment2}
	\end{center}
	\caption{Experiment Middle \label{fig:experiment2}}
\end{figure}

\begin{figure}[H]
	\begin{center}
		\includegraphics[width=8cm]{figures/Experiment3}
	\end{center}
	\caption{Experiment End \label{fig:experiment3}}
\end{figure}

The first area of comparison is the effect of the algorithms on the number of perimeter agents. The baseline swarm's agents oscillates but remain in a relatively stable state with a constant number of perimeter agents and the internal anomaly persists~(Fig.~\ref{fig:baselineSwarm}). The maximum and minimum number of perimeter agents is shown in table~\ref{tab:perimeterLimits}. 

\begin{figure}[H]
	\begin{center}
		\includegraphics[width=8cm]{figures/PerimeterCount}
	\end{center}
	\caption{Perimeter Count of baseline, gap reduction and perimeter compression. \label{fig:perimeterCount}}
\end{figure}

\begin{table}[h]
	\centering
	\tiny
	\begin{tabular}{|c|c|r|r|r|r|r|r|r|}
		\hline
		\rowcolor[HTML]{000000} 
		{\color[HTML]{FFFFFF} Comp.} &{\color[HTML]{FFFFFF} Base} & {\color[HTML]{FFFFFF} Void} & {\color[HTML]{FFFFFF} 1} & {\color[HTML]{FFFFFF} 2} & {\color[HTML]{FFFFFF} 3} & {\color[HTML]{FFFFFF} 4} & {\color[HTML]{FFFFFF} 5} & {\color[HTML]{FFFFFF} 6}\\ \hline
		Max & \texttt{75} & \texttt{90} &\texttt{90} & \texttt{90} & \texttt{94} & \texttt{92} & \texttt{95} & \texttt{93} \\ \hline
		Min & \texttt{75} & \texttt{51} &\texttt{51}  & \texttt{51} & \texttt{51} & \texttt{49} & \texttt{49} & \texttt{49}\\ \hline
		Mean & \texttt{75} & \texttt{62} &\texttt{59}  & \texttt{58} & \texttt{60} & \texttt{59} & \texttt{57} & \texttt{59}\\ \hline
		Std & \texttt{0} & \texttt{6} &\texttt{9}  & \texttt{10} & \texttt{10} & \texttt{8} & \texttt{10} & \texttt{8}\\ \hline
	\end{tabular}
	\caption{Perimeter agents}
	\label{tab:perimeterLimits}
\end{table}

\begin{figure}[H]
	\begin{center}
		\includegraphics[width=8cm]{figures/PerimeterCount2}
	\end{center}
	\caption{Perimeter Count of baseline, gap reduction and Experiment 6. \label{fig:perimeterCount2}}
\end{figure}

\begin{figure}[H]
	\begin{center}
		\includegraphics[width=8cm]{figures/DistanceMetric1}
	\end{center}
	\caption{Distance metric of baseline, gap reduction and perimeter compression. \label{fig:distanceMetric}}
\end{figure}

\begin{figure}[H]
	\begin{center}
		\includegraphics[width=8cm]{figures/InteragentCohesionMean}
	\end{center}
	\caption{Inter-agent Cohesion Mean. \label{fig:interagentCohesionMean}}
\end{figure}

\begin{figure}[H]
	\begin{center}
		\includegraphics[width=8cm]{figures/InteragentCohesionStd}
	\end{center}
	\caption{Inter-agent Cohesion Std. \label{fig:interagentCohesionStd}}
\end{figure}

\begin{figure}[H]
	\begin{center}
		\includegraphics[width=8cm]{figures/InteragentRepulsionMean}
	\end{center}
	\caption{Inter-agent Repulsion Mean. \label{fig:interagentRepulsionMean}}
\end{figure}

\begin{figure}[H]
	\begin{center}
		\includegraphics[width=8cm]{figures/InteragentRepulsionStd}
	\end{center}
	\caption{Inter-agent Repulsion Std. \label{fig:interagentRepulsionStd}}
\end{figure}


\begin{figure}[H]
	\begin{center}
		\includegraphics[width=8cm]{figures/baselineSwarmPerimeter}
	\end{center}
	\caption{Baseline swarm in stabilised configuration. \label{fig:baselineSwarmPerimeter}}
\end{figure}

\subsection{Gap compression}
\subsection{Perimeter compression}

Compression 1

\begin{figure}[H]
	\begin{center}
		\includegraphics[width=5cm]{figures/baselineSwarm1}
	\end{center}
	\caption{Baseline swarm in with compression set 1 resultant configuration. \label{fig:baselineSwarm1}}
\end{figure}

\begin{figure}[H]
	\begin{center}
		\includegraphics[width=8cm]{figures/baselineSwarmMagnitude1}
	\end{center}
	\caption{Baseline swarm in with compression set 1. \label{fig:baselineSwarmMagnitude1}}
\end{figure}

\begin{figure}[H]
	\begin{center}
		\includegraphics[width=8cm]{figures/baselineSwarmPerimeter1}
	\end{center}
	\caption{Baseline swarm in with compression set 1. \label{fig:baselineSwarmPerimeter1}}
\end{figure}

\begin{figure}[H]
	\begin{center}
		\includegraphics[width=9cm]{figures/Figure_1}
	\end{center}
	\caption{Sample}
\end{figure}
\begin{figure}[H]
	\begin{center}
		\includegraphics[width=9cm]{figures/Figure_2}
	\end{center}
	\caption{Sample}
\end{figure}
\begin{figure}[H]
	\begin{center}
		\includegraphics[width=9cm]{figures/Figure_3}
	\end{center}
	\caption{Sample}
\end{figure}
\begin{figure}[H]
	\begin{center}
		\includegraphics[width=9cm]{figures/Figure_4}
	\end{center}
	\caption{Sample}
\end{figure}
\begin{figure}[H]
	\begin{center}
		\includegraphics[width=9cm]{figures/Figure_5}
	\end{center}
	\caption{Sample}
\end{figure}
\begin{figure}[H]
	\begin{center}
		\includegraphics[width=9cm]{figures/Figure_6}
	\end{center}
	\caption{Sample}
\end{figure}
\begin{figure}[H]
	\begin{center}
		\includegraphics[width=9cm]{figures/Figure_7}
	\end{center}
	\caption{Sample}
\end{figure}
\begin{figure}[H]
	\begin{center}
		\includegraphics[width=9cm]{figures/Figure_8}
	\end{center}
	\caption{Sample}
\end{figure}
\begin{figure}[H]
	\begin{center}
		\includegraphics[width=9cm]{figures/Figure_9}
	\end{center}
	\caption{Sample}
\end{figure}
\begin{figure}[H]
	\begin{center}
		\includegraphics[width=9cm]{figures/Figure_10}
	\end{center}
	\caption{Sample}
\end{figure}

\begin{figure}[H]
	\begin{center}
		\includegraphics[width=9cm]{figures/mag-pr-0.9-mean}
	\end{center}
	\caption{MAG pr=0.9 mean}
\end{figure}

\begin{figure}[H]
	\begin{center}
		\includegraphics[width=9cm]{figures/mag-pr-0.9-std}
	\end{center}
	\caption{MAG pr=0.9 std}
\end{figure}

\begin{figure}[H]
	\begin{center}
		\includegraphics[width=9cm]{figures/mag-pr-0.9-pc-100-error}
	\end{center}
	\caption{MAG pr=0.9 pc=100 error}
\end{figure}

\begin{figure}[H]
	\begin{center}
		\includegraphics[width=9cm]{figures/dist-pr-0.9-mean}
	\end{center}
	\caption{DIST pr=0.9 mean}
\end{figure}

\begin{figure}[H]
	\begin{center}
		\includegraphics[width=9cm]{figures/dist-pr-0.9-std}
	\end{center}
	\caption{DIST pr=0.9 std}
\end{figure}

\begin{figure}[H]
	\begin{center}
		\includegraphics[width=9cm]{figures/dist-pr-0.9-pc-100-error}
	\end{center}
	\caption{DIST pr=0.9 pc=100 error }
\end{figure}


\subsection{Comparison}

\section{Conclusions}\label{conclusions}
From the initial simulations it is possible to show that the technique is able to successfully remove voids and surround an obstacle as shown in the video \href{https://youtu.be/3eY1vvq0JWo}{https://youtu.be/3eY1vvq0JWo}.

\section{Future Work}

\bibliographystyle{abbrv}
\bibliography{perimeter}

\end{document}