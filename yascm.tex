\documentclass[12pt,a4paper]{article}
\usepackage{graphicx, hyperref, relsize, amsmath, amssymb}

\renewcommand{\vec}[1]{\overrightarrow{#1}}
\newcommand{\pangle}{\mathit{\alpha}}
\newcommand{\leqaz}[3]{#2 \leq_{\pangle_#1} #3}
\newcommand{\angleordered}[2]{\langle #2 \rangle_{\leqaz{#1}{}{}}}
\newcommand{\prm}{\mathsf{prm}}
\newcommand{\kc}{\mathit{k_c}}
\newcommand{\kr}{\mathit{k_r}}
\newcommand{\rb}{\mathit{R}}

\title{Just the bare bones of the simplified model}
\author{David}
\date{\today}


\begin{document}
\maketitle

\section{Swarms, agents, cohesion neighbours and perimeter}
A swarm $S$ comprises a set of \emph{agents}, $b$, $b'$, $b''$, $b_0$, $b_1$, etc.
An agent is modelled simply as a point in the 2-D Euclidean plane, specified by
a position vector in some coordinate system. Notice that, by definition, two
different agents cannot occupy the same position.

Assume a global constant, $C$, associated with a swarm, that
determines the radius of the \emph{cohesion field} of each agent in the swarm. 

For each agent, $b \in S$, its \emph{cohesion neighbours}, is the set of
agents, $n_c(b)$, defined by 
\begin{equation}\label{eq:c_nbrs}
n_c(b) = \{b' \in S~:~b' \neq b \land \|b' - b\| \leq C\}
\end{equation}

It is useful to define an ordering on an agent's cohesion neighbours.  We
choose to order the cohesion neighbours of an agent $b$ by their \emph{polar
angle} with respect to $b$. The polar angle with respect to $b$ of $b'$,
$\pangle(b, b')$, is the counterclockwise angle that vector $\vec{bb'} = b' -
b$ makes with the positive x axis:
\begin{equation}\label{eq:pangle}
	\pangle(b, b') = \mathsf{atan2}((b'-b)_y, (b'-b)_x)
\end{equation} 

A partial ordering of agents by polar angle with respect to a specific agent,
$b$, is denoted $\leqaz{b}{}{}$, and is defined by: 
\begin{equation}\label{angle_ordering}
	\leqaz{b}{b'}{b''} \iff \pangle(b, b') \leq \pangle(b, b'')
\end{equation}

We denote by $\angleordered{b}{b_0, b_1, .., b_{n-1}}$ a bijection from $\{0,
.., n-1\} \rightarrow n_c(b)$ that is ordered by polar angle, i.e. $\forall i,
j~:~0 \leq i, j, < n \cdot i \leq j \implies \leqaz{b}{b_i}{b_j}$.

In this paper, we propose that the behaviour of an agent should be modified
depending on whether or not it is on a \emph{perimeter}. An agent $b$ is on a
perimeter if it satisfies any one of three conditions:
\begin{enumerate}
	\item consecutive neighbours are not within each other's cohesion field, or
	\item consecutive neighbours subtend a reflex angle, or
	\item the agent has too few neighbours.
\end{enumerate}
A function, $\prm(b)$, specifies these conditions formally. Let $b$ be the
agent of interest and $b'$, $b''$ any pair of consecutive neighbours of $b$ in
the angle-sorted list $\angleordered{b}{b_0, b_1, .., b_{n-1}}$, i.e. $b' =
b_i, b'' = b_{(i+1)\%n}$ for some $i \in \{0,..,n-1\}$.  Then $\prm(b)$ iff any
one of the following conditions is satisfied:
\begin{enumerate}
\item $b' \notin n_c(b'')$,
\item $\delta > \pi$, where $\delta = \pangle(b, b'') - \pangle(b, b')$ (or $\delta = \pangle(b, b'') - \pangle(b, b') + 2\pi$ if the former is negative), or
\item $n_c(b) < 3$.
\end{enumerate}



\section{Cohesion vector}
\begin{equation}\label{eq:coh2}
	v_c(b) = \frac{1}{|n_c(b)|} \sum_{b' \in n_c(b)} \kc[p_b, p_{b'}] (b' - b)
\end{equation}
where $|n_c(b)|$ denotes the cardinality of $n_c(b)$, $p_b = \prm(b)$, $p_{b'} 
= \prm(b')$, and 
$\kc$ is a 2x2 boolean-indexed array of constants that determine the weight
of a component of the cohesion vector according to
whether the interaction between $b,b'$ is between non-perimeter agents,
non-perimeter--perimeter, perimeter--non-perimeter, or perimeter--perimeter
agents.


\section{Repulsion vector}
The set of repellers of $b$ is
\begin{equation}\label{eq:rep1}
	n_r(b) = \{b' \in \mathcal{S} : b \neq b' \wedge \|b' - b\| \leq \rb[p_b,p_{b'}]\}
\end{equation}
where $p_b = \prm(b)$, $p_{b'} = \prm(b')$, and $\rb$ is a 2x2 boolean-indexed
array of constants that determine the radius of the \emph{repulsion field} for
agents in the swarm, according to whether the interaction between $b,b'$ is
between non-perimeter agents, non-perimeter--perimeter,
perimeter--non-perimeter, or perimeter--perimeter agents.

Now $v_r(b)$ is defined by
\begin{equation}\label{eq:rep2}
	v_r(b) = \frac{1}{|n_r(b)|}\sum_{b' \in n_r(b)} \kr[p_b,p_{b'}] \left(1 - \frac{\rb[p_b,p_{b'}]}{\|b'-b\|} \, \right) (b'-b)
\end{equation}
where $p_b = \prm(b)$, $p_{b'} = \prm(b')$, and $\kr$ is a 2x2 boolean-indexed
array of constants that determine the weight of a component of the repulsion
vector according to whether the interaction between $b,b'$ is between
non-perimeter agents, non-perimeter--perimeter, perimeter--non-perimeter, or
perimeter--perimeter agents.

\section{Gap-filling vector}
In addition to cohesion and repulsion vectors, a \emph{gap-filling} vector can
also be used to contribute to agent behaviour. Gap-filling vectors have proven
useful in quickly reducing internal voids and in controlling the shape of the
external perimeter.

A gap-filling vector for $b$ contributes a motion of $b$ towards the midpoint
of a gap identified in the perimeter test for $b$.

Let $\angleordered{b}{b_0, b_1, .., b_{n-1}}$ be the cohesion neighbours of $b$
in polar angle order, and let $b' = b_i$  and $b'' = b_{(i+1)\%n}$ be the first
pair of consecutive neighbours that satisfy either condition (1) or condition
(2) of the perimeter function $\prm()$, then the gap-filling vector, $v_g(b)$,
for agent $b$ is defined
\begin{equation}\label{eq:gap}
v_g(b) = k_g \left (\frac{b' + b''}{2} - b \right) = k_g \frac{\vec{bb'} + \vec{bb''}}{2} 
\end{equation}
If there is no such pair of consecutive neighbours then $v_g(b) = 0$.

$k_g$ is a weighting for the gap-filling vector allowing the combination of it
with the other motion vectors (cohesion, repulsion, ...) to be ``tuned''.

A stricter alternative to this is to choose the first consecutive neighbour
pair $b',b''$ that satisfy condition (1), ignoring condition (2).  Again,
$v_g(b)$ is defined by eq (\ref{eq:gap}) if such a pair exists, or 0 otherwise.

\section{Resultant vector}
The resultant vector is simply the sum of the cohesion, repulsion and
gap-filling vectors:
\begin{equation}\label{eq:res}
	v(b) = v_c(b) + v_r(b) + v_g(b) 
\end{equation}

\end{document}
